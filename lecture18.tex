\documentclass{article}

\usepackage{url}
\usepackage{amsmath,amssymb}
\usepackage{graphicx,svg}

\begin{document}
\title{Lecture 18\\Inverse Z Transform}
\author{C.L. Wyatt}
\date{\today}
\maketitle

Consider a causal signal $f[n]$ that may, or may not be absolutely summable. Let $g[n]=(r)^{n} f[n]$ for some $r>0$ such that

$$
\sum_{n=0}^{\infty}|g[n]|=\sum_{n=0}^{\infty}\left|r^{n} f[n]\right|<\infty
$$

Then the DTFT of g[n] exists and is given by

$$
\begin{aligned}
& G\left(e^{j \omega}\right)=\sum_{n=-\infty}^{\infty} g[n] e^{-j \omega n}=\sum_{n=0}^{\infty} g[n] e^{-j \omega n} \\
&=\sum_{n=0}^{\infty} r^{n} f[n] e^{-j \omega n} \\
&=\sum_{n=0}^{\infty} f[n]\left(r^{-1} e^{j \omega}\right)^{-n} \\
&=\left.\mathcal{Z}_{1}\{f[n]\}\right|_{z=r^{-1} e^{j \omega}} 
\end{aligned}
$$
for $r$ fixed so that $|z| = |r^{-1} e^{j\omega}| = |r^{-1}| \in$ ROC.

The inverse DTFT is then

$$
g[n]=r^{n} f[n]=\frac{1}{2 \pi} \int_{-\pi}^{\pi} G\left(e^{j \omega}\right) e^{j \omega n} d \omega
$$

multiply through by $r^{-n}$

$$
r^{-n} g[n]=f[n]=\frac{1}{2 \pi} \int_{\pi}^{\pi} G\left(e^{j n}\right)\left(r^{-1} e^{j \omega}\right)^{n} d \omega
$$

Substitute for $G\left(e^{j \omega}\right)$

$$
f[n]=\frac{1}{2 \pi} \int_{-\pi}^{\pi} \left[ \sum_{n=0}^{\infty} f[n]\left(r^{-1} e^{j \omega}\right)^{-n} \right] \left(r^{-1} e^{j \omega}\right)^{n} d \omega
$$

Let $z=r^{-1}e^{j\omega}$. Then

$$
f[n]=\left. \frac{1}{2 \pi} \int_{-\pi}^{\pi} F(z) \left(r^{-1} e^{j \omega}\right)^{n} d \omega \quad \right|_{z=r^{-1}e^{j\omega}}
$$

Since $r$ is a constant $dz=jr^{-1}e^{j\omega}\; d\omega$ and $d\omega = \frac{1}{j} re^{-j\omega}\; dz$ and

$$
f[n]=\frac{1}{2 \pi} \oint_C F(z) \left(r^{-1} e^{j \omega}\right)^{n} \frac{1}{j} re^{-j\omega}\; dz
$$
where $C$ is the contour $z=r^{-1}e^{j\omega}$ for $\omega\in[-\pi, \pi]$, i.e. a counter-clockwise circle of radius $r^{-1}$ in the ROC. This circle will necessarily enclose any singularities of $F(z)$ due to causality.

Noting

$$
z^{n-1} = z^n z^{-1} = \left(r^{-1}e^{j\omega} \right)^n \, \left(r^{-1}e^{j\omega} \right)^{-1} = \left(r^{-1}e^{j\omega} \right)^n \, \left(re^{-j\omega} \right) 
$$

We can write this as

$$
f[n]= \frac{1}{2 \pi j} \oint_C F(z) z^{n-1}\; dz
$$

a complex integral defining the inverse unilateral Z transform. We can solve this integral using the method of residues.

Notice this is straightforward compared to the inverse Laplace transform since we do not need the limit of the Bromwitch contour.

Example 1: Recall that $\mathcal{Z}_1\left\{(\gamma)^n\, u[n]\right\} = \frac{z}{z-\gamma}$ for $|z| > |\gamma|$. Let's look at the inverse. For $n \geq 0$:

$$
\begin{aligned}
  x[n] &= \mathcal{Z}_1^{-1}\left\{\frac{z}{z-\gamma}\right\}\\
  &= \frac{1}{2 \pi j} \oint_C  \frac{z}{z-\gamma}  z^{n-1}\; dz\\
  &= \frac{1}{2 \pi j} \oint_C  \frac{z^n}{z-\gamma}\; dz\\
  &= \frac{1}{2 \pi j} 2 \pi j [K_1] 
\end{aligned}
$$

where $K_1$ is the residue

$$
K_1 = \left. (z-\gamma)\frac{z^n}{z-\gamma}\right|_{z = \gamma} = \gamma^n
$$
Thus, as expected, the inverse is $x[n] = \gamma^n\, u[n]$ for all $n$ since the ROC corresponds to a causal signal.

What about the bilateral inverse z-transform? We will take an approach similar to that in the Laplace transform, first writing a non-causal signal as an anti-causal plus causal component.

$$
x[n] = \underbrace{x[n] u[-n-1]}_{\text{anti-causal part } x_a[n]} +  \underbrace{x[n] u[n]}_{\text{causal part } x_c[n]} 
$$

The bilateral Z transform is

$$
\begin{aligned}
  \mathcal{Z}_2\left\{x[n]\right\} &= \sum\limits_{n = -\infty}^{\infty} x[n] z^{-n}\\
  &= \sum\limits_{n = -\infty}^{-1} x_a[n] z^{-n} + \sum\limits_{n = 0}^{\infty} x_c[n] z^{-n}\\
  &= \sum\limits_{n = 1}^{\infty} x_a[-n] z^{n} + \mathcal{Z}_1\left\{x_c[n]\right\}\\
  &= \sum\limits_{n = 0}^{\infty} x_a[-n] z^{n} - x_a[0] z^{0} + \mathcal{Z}_1\left\{x_c[n]\right\}\\
  &= \left. \mathcal{Z}_1\left\{x_a[-n]\right\}\right|_{z \rightarrow z^{-1}} + \mathcal{Z}_1\left\{x_c[n]\right\}
\end{aligned}
$$
since $x_a[0] z^{0} = x_a[0] = 0$ by definition. The ROC for the anti-causal part is of the form $|z| < \gamma_a$. The ROC for the causal part is of the form $|z| > \gamma_c$. Thus the ROC of the entire transform is $\gamma_c < |z| < \gamma_a$ if it is not the empty set. This forms a ring in the complex plane.

Example 2:

$$
\begin{aligned}
  \mathcal{Z}_2\left\{ -\gamma^n\; u[-n-1]\right\} &= \sum\limits_{n = -\infty}^{-1} -\gamma^n\, z^{-n}\\
  &= \sum\limits_{n = 1}^{\infty} -\gamma^{-n}\, z^{n}\\
  &= -\sum\limits_{n = 1}^{\infty} \left(\gamma^{-1}\, z\right)^n\\
  &= - \lim_{N\rightarrow\infty} \frac{\left(\gamma^{-1}\, z\right)^{N+1} - \left(\gamma^{-1}\, z\right)^{1}}{\left(\gamma^{-1}\, z\right) - 1}\\
  &= - \frac{ - \left(\gamma^{-1}\, z\right)^{1}}{\left(\gamma^{-1}\, z\right) - 1}\\
\end{aligned}
$$
if $|\gamma^{-1} z| < 1$ or $|z| < |\gamma|$. Simplifying a bit we get

$$
\frac{z}{z-\gamma} \text{ for } |z| < |\gamma|
$$

Notice this is the same function of z as the Z transform of the causal signal $\gamma^n\, u[n]$, but the ROC is different. Thus like for Laplace the only way to tell causal from anti-causal components is via the singularity locations relative to the ROC.

If the singularity lies outside the ROC ring, then it corresponds to an anti-causal component. If the singularity lies inside the ROC ring, then it corresponds to a causal component. 

Example 3: an example with complex roots.

$$
X(z) = \frac{z}{z^2 + 1} = \frac{z}{(z+j)(z-j)} \text{ for } |z| > 1
$$

This is a causal ROC so for $n \geq 0$

$$
x[n] = \frac{1}{2\pi j} \oint_C X(z) z^{n-1}\; dz
$$

where $C$ is the circle with radius greater than one.

$$
\begin{aligned}
  x[n] &= \frac{1}{2\pi j} \oint_C \frac{z^n}{(z+j)(z-j)}\; dz\\
  &= \frac{1}{2\pi j} 2\pi j[K_1 + K_2] 
\end{aligned}
$$

where the residues are

$$
K_1 = \left. \frac{z^n}{(z-j)} \right|_{z = -j} = \frac{(-j)^n}{(-j-j)} = -\frac{1}{2j} \left(e^{-j\frac{\pi}{2}}\right)^n 
$$

$$
K_2 = \left. \frac{z^n}{(z+j)} \right|_{z = j} = \frac{(j)^n}{(j+j)} = \frac{1}{2j} \left(e^{j\frac{\pi}{2}}\right)^n 
$$

Substituting we get

$$
\begin{aligned}
  x[n] &= \frac{1}{2j} \left(e^{j\frac{\pi}{2}}\right)^n -\frac{1}{2j} \left(e^{-j\frac{\pi}{2}}\right)^n\\
  &= \sin\left(\frac{\pi}{2}n \right) \text{ for } n \geq 0
\end{aligned}
$$

Thus for all $n$, $x[n] = \sin\left(\frac{\pi}{2}n \right) u[n]$.

Example 4: Given the transfer function

$$
H(z) = \frac{z+a}{z+b} \text{ for } |z| > |b| \text{ and } a,b\in\mathbb{R}
$$

find the corresponding impulse response of the system, $h[n]$.

\textbf{Solution:} Since the ROC corresponds to a causal signal, for $n \geq 0$

$$
h[n] = \frac{1}{2\pi j} \oint_C H(z) z^{n-1}\; dz
$$

where $C$ is the circle with radius greater than $|b|$.

$$
\begin{aligned}
  h[n] &= \frac{1}{2\pi j} \oint_C \frac{z+a}{z+b} z^{n-1}\; dz\\
  &= \frac{1}{2\pi j} \oint_C \frac{(z+a)z^n}{z(z+b)}\; dz\\
  &= \frac{1}{2\pi j} 2\pi j [K_1 + K_2]
\end{aligned}
$$

where

$$
K_1 = \left. \frac{(z+a)z^n}{z+b} \right|_{z=0} = \frac{a}{b}\delta[n]
$$

$$
K_2 = \left. \frac{(z+a)z^n}{z} \right|_{z=-b} = \frac{b-a}{b} \left(-b\right)^n
$$

Thus for all $n$

$$
h[n] = \frac{a}{b}\delta[n] + \frac{b-a}{b} \left(-b\right)^n\, u[n] 
$$

Example 5: Consider the LTI DT system with impulse response $h[n] = \left(\frac{1}{4}\right)^n\, u[n]$ and input $x[n] = \cos\left(\frac{\pi}{4} n\right)\, u[n]$. What is ths Z-transform of the output and the time domain representation $y[n]$?




\end{document}
