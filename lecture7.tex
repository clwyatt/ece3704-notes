\documentclass{article}

\usepackage{url}
\usepackage{amsmath,amssymb}
\usepackage{graphicx,svg}

\begin{document}
\title{Lecture 7\\ Complex Analysis II}
\author{C.L. Wyatt}
\date{\today}
\maketitle

In today's lecture we continue our introduction to complex analysis by defining analytic complex functions, the complex derivative, and looking closer at rational complex functions.

\section{Complex Derivatives}

A complex function $f(s)$ is \textit{analytic} in some domain $R\subset\mathbb{C}$ if the function is

\begin{itemize}
\item single valued, and
\item has a finite complex derivative $f^\prime(s) \equiv \frac{df}{ds}$ for all $s\in R$.
\end{itemize}

Let $f(s) = f(x + jy) = u(x,y) + jv(x,y)$. The function has a complex derivative if and only if

\[
\frac{\partial u}{\partial x} = \frac{\partial v}{\partial y}  
\]
and
\[
\frac{\partial u}{\partial y} = -\frac{\partial v}{\partial x}  
\]
These are called the Cauchy-Riemann conditions. If these conditions are met then the complex derivative is given by

\[
f^\prime(s) = \frac{\partial u}{\partial x} + j \frac{\partial v}{\partial x} = \frac{\partial v}{\partial y} - j \frac{\partial u}{\partial y}
\]

If a complex function is analytic over the entire complex plane, i.e. $R = \mathbb{C}$, it is called an \textit{entire} function.

If a complex function is not analytic at a point $s_0$, but is analytic in the neighborhood of $s_0$, then the point is called a \textit{singularity} or a \textit{singular point} of the function.

\subsection{Examples}

\textbf{Example 1:} Let $s = x+jy$ and define

\[
f(s) = e^s = f(x+jy) = e^{x+jy} = e^x e^{jy}
\]
Using Euler's relation

\[
f(x+jy) = \underbrace{e^x\cos(y)}_{u(x,y)} + j\underbrace{e^x \sin(y)}_{v(x,y)}
\]
Checking the Cauchy-Riemann conditions

\[
\frac{\partial u}{\partial x} = e^x\cos(y)
\]

\[
\frac{\partial u}{\partial y} = -e^x\sin(y)
\]

\[
\frac{\partial v}{\partial x} = e^x\sin(y)
\]
\[
\frac{\partial v}{\partial y} = e^x\cos(y)
\]
Thus $\frac{\partial u}{\partial x} = \frac{\partial v}{\partial y} = e^x\cos(y)$ and $\frac{\partial u}{\partial y} = -\frac{\partial v}{\partial x} = -e^x\sin(y)$ and the function is analytic everywhere and thus an entire function. The complex derivative is

\[
f^\prime(s) = e^x\cos(y) + je^x\sin(y) = e^x e^{jy} = e^{x+jy} = e^s
\]
just as if $s$ was real and $f$ a real function.

\textbf{Example 2:} Let $f(s) = \frac{1}{s}$, for $|s| > 0$, i.e. $s \neq 0$. Then

\[
f(s) = f(x+jy) = \frac{1}{x+jy} = \frac{1}{x+jy} \cdot \frac{x-jy}{x-jy} = \frac{x-jy}{x^2 + y^2} = \underbrace{\frac{x}{x^2 + y^2}}_{u(x,y)} + j \underbrace{\frac{-y}{x^2 + y^2}}_{v(x,y)}
\]

Checking the Cauchy-Riemann conditions

\[
\frac{\partial u}{\partial x} = \frac{-x^2 + y^2}{(x^2 + y^2)^2}
\]

\[
\frac{\partial u}{\partial y} = \frac{-2xy}{(x^2 + y^2)^2}
\]

\[
\frac{\partial v}{\partial x} = \frac{2xy}{(x^2 + y^2)^2}
\]

\[
\frac{\partial v}{\partial y} = \frac{-x^2 + y^2}{(x^2 + y^2)^2}
\]

we we see that the function is analytic for $s \in \mathbb{C} - (0 + j0)$ and

\[
f^\prime(s) = \frac{-x^2 + y^2}{(x^2 + y^2)^2} + j \frac{2xy}{(x^2 + y^2)^2}
\]

With some work you can show this is the same as

\[
f^\prime(s) = -\frac{1}{s^2}
\]

again, just as if $s$ was real and $f$ a real function.

\textbf{Example 3:} Now for a counter-example. Consider

\[
f(s) = s^* = f(x + jy) = x - jy
\]
Thus $u(x,y) = x$ and $v(x,y) = -y$. Checking the Cauchy-Riemann conditions,

\[
\frac{\partial u}{\partial x} = 1
\]

\[
\frac{\partial u}{\partial y} = 0
\]

\[
\frac{\partial v}{\partial x} = 0
\]

\[
\frac{\partial v}{\partial y} = -1
\]

we see $\frac{\partial u}{\partial x} = 1 \neq \frac{\partial v}{\partial y} = -1$ thus the function is not analytic and no derivative exists.

\textbf{Example 4:} Show that $s_0 = -1$ is a singularity of an analytic complex function
\[
f(s) = \frac{1}{s+1} = f(x +jy) = \frac{1}{(1+x) + jy}
\]
Rearranging to get the real part and imaginary part

\[
\frac{1}{(1+x) + jy}\frac{(1+x) - jy}{(1+x) - jy} = \underbrace{\frac{1+x}{(1+x)^2 + y^2}}_{u(x,y)} + j \underbrace{\frac{-y}{(1+x)^2 + y^2}}_{v(x,y)}
\]

Checking the Cauchy-Riemann conditions,

\[
\frac{\partial u}{\partial x} = \frac{-(1+x)^2 + y^2}{[(1+x)^2 + y^2]^2}
\]

\[
\frac{\partial u}{\partial y} = \frac{-2(1+x)y}{[(1+x)^2 + y^2]^2}
\]

\[
\frac{\partial v}{\partial x} = \frac{2(1+x)y}{[(1+x)^2 + y^2]^2}
\]

\[
\frac{\partial v}{\partial y} = \frac{-(1+x)^2 + y^2}{[(1+x)^2 + y^2]^2}
\]

we see that $\frac{\partial u}{\partial x} = \frac{\partial v}{\partial y}$ and $\frac{\partial u}{\partial y} = -\frac{\partial v}{\partial x}$ except at $x=-1, y = 0$ where

\[
\frac{-(1+x)^2 + y^2}{[(1+x)^2 + y^2]^2} = \frac{0}{0} = \text{undefined}
\]

Thus the function is analytic except at $s_0 = -1 + j0 = -1$.

\section{Rational Complex Functions}

The previous example is a \textit{rational function}, a ratio of polynomials in $s$. These will be very important later in the course. We will need to be very adept at doing manipulations of such functions. The general case can be written as

\[
f(s) = \frac{\sum\limits_{k=0}^M b_k s^k}{\sum\limits_{k=0}^N a_k s^k} \equiv \frac{P(s)}{Q(s)}
\]

\begin{itemize}
\item If $M < N$ the ratio is said to be \textit{strictly proper}.
\item If $M \leq N$ the ratio is \textit{proper}.
\item If $M > N$ the ratio is \textit{improper}.
\end{itemize}

Example: $M = 2, N=3$

\[
f(s) = \frac{b_0 + b_1 s + b_2 s^2}{a_0 + a_1 s + a_2 s^2 + a_3 s^3}
\]
is a strictly proper rational function.


The roots of the denominator polynomial $Q(s)$ are the singularities of $f$.

\subsection{Partial Fraction Expansion}

We will often be interested in writing rational functions as a \textit{partial fraction expansion} of the factors of the denominator. A ratio of polynomials that is improper can be written as the sum of a polynomial and a strictly proper rational function. Example

\[
f(s) = \frac{2s^3 + 9s^2 + 11s + 2}{s^2 + 4s + 3} = (2s+1) + \frac{s-1}{s^2 + 4s + 3}
\]

Once a rational function is proper we can expand it in terms of the factors of $Q(s)$. Example

\[
f(s) = \frac{1}{s^2 + 3s +2} = \frac{1}{(s+1)(s+2)} = \frac{A}{s+1} + \frac{B}{s+2}
\]

To find $A$ and $B$ we can "clear" the fractions

\[
\frac{A}{s+1} + \frac{B}{s+2} = \frac{1}{(s+1)(s+2)} = \frac{A(s+2) + B(s+1)}{(s+1)(s+2)}
\]
which gives the equation for the numerator
\[
A(s+2) + B(s+1) = 1 + 0s
\]
which requires $A = 1$ and $B = -1$, so that
\[
f(s) = \frac{1}{s+1} + \frac{-1}{s+2}
\]

Clearing fractions can be tedious with higher order systems. A shortcut is to use the Heaviside "cover-up" method, or finding the \textit{residues}.

\textbf{Case \#1:} non-repeated singularities
\[
f(s) = \frac{P(s)}{(s-\alpha_1)(s-\alpha_2)\cdots (s-\alpha_N)}
\]
where $\alpha_i$ are the distinct singularities (roots of $Q(s)$). Then
\[
f(s) = \frac{K_1}{s-\alpha_1} + \frac{K_1}{s-\alpha_1} + \cdots + \frac{K_N}{s-\alpha_N} 
\]

To find the $K_i$ values we multiply through by that term ($s-\alpha_i$) and evaluate the result when $s = \alpha_i$
\[
K_i = \left. (s-\alpha_i)f(s) \right|_{s = \alpha_i} 
\]

Example:

\[
f(s) = \frac{s^2 + 2s + 7}{(s+1)(s-3)(s+5)} = \frac{K_1}{s+1} + \frac{K_2}{s-3} + \frac{K_3}{s+5}
\]

We note that $M = 2$ and $N = 3$ so that the function is strictly proper. To find the constants

\[
K_1 = \left. \frac{s^2 + 2s + 7}{(s-3)(s+5)} \right|_{s = -1} = -\frac{3}{8}
\]

\[
K_2 = \left. \frac{s^2 + 2s + 7}{(s+1)(s+5)} \right|_{s = 3} = \frac{11}{16}
\]

\[
K_3 = \left. \frac{s^2 + 2s + 7}{(s+1)(s-3)} \right|_{s = -5} = \frac{11}{16}
\]

This works even if the roots are complex. For example:

\[
f(s) = \frac{1}{(s + 1 + j)(s + 1 -j)} = \frac{K_1}{s + 1 + j} + \frac{K_2}{s + 1 -j}
\]

\[
K_1 = \left. \frac{1}{s + 1 - j} \right|_{s = -1-j} = -\frac{1}{2j}
\]

\[
K_2 = \left. \frac{1}{s + 1 + j} \right|_{s = -1+j} = \frac{1}{2j}
\]

\textbf{Case \#2:} However we will often want to avoid working with complex roots directly. Another approach is to combine them into a quadratic term. Here is an example:

\[
f(s) = \frac{s+2}{s^3 + 3s^2 + 4s + 2} = \frac{s+2}{(s+1)(s+1 +j)(s +1 -j)} = \frac{s+2}{(s+1)(s^2 +2s + 2)} = \frac{A}{s+1} + \frac{Bs + C}{s^2 + 2s + 2}
\]

To find $A$ we use the same residue method

\[
A = \left. \frac{s+2}{s^2 + 2s + 2} \right|_{s = -1} = 1
\]

To find $B$ and $C$ we can clear fractions or use another shortcut. To find $C$, let $s = 0$. We get

\[
\frac{A}{1} + \frac{C}{2} = 1
\]
which implies that $C = 0$. To find $B$ multiply through by $s$ and let $s\rightarrow \infty$.

\[
\frac{As}{s+1} + \frac{Bs^2 + Cs}{s^2 + 2s + 2} = \frac{s^2 + 2s}{s^3 + 3s^2 + 4s + 2}
\]

In the first term divide top and bottom by $s$, in the second term by $s^2$, and on the right-hand side by $s^3$:


\[
\frac{A}{1+s^{-1}} + \frac{B + Cs^{-1}}{1 + 2s^{-1} + 2s^{-2}} = \frac{s^{-1} + 2s^{-2}}{1 + 3s^{-1} + 4s^{-2} + 2s^{-3}}
\]

Now let $s\rightarrow \infty$. We get

\[
\frac{A}{1} + \frac{B}{1} = \frac{0}{1} = 0
\]
which implies $B = -1$.

Or you can just clear the fractions and substitute for $A$ and $C$.

\begin{align}
  As^2 + 2As + 2A + Bs^2 + Cs + Bs + C &= s+2\\
  (A+B)s^2 + (2A + B + c)s + 2A +C = s+2\\
\end{align}
We only need one equation, so lets use $2A + B + C = 1$. Solving for $B$ and substituting for $A.C$ gives $B = -1$ as above.

Thus the final result is

\[
f(s) = \frac{1}{s+1} + \frac{-s}{s^2 + 2s + 2}
\]

\textbf{Case \# 3:} One other complication is when we have repeated roots (a root $\lambda$, repeated $r > 1$ times).

\[
f(s) \frac{P(s)}{(s-\lambda)^r\, (s-\alpha_1)(s-\alpha_2)\cdots(s-\alpha_{N-r})}
\]

To handle this case we expand the ratio as

\[
f(s) = \frac{a_0}{(s-\lambda)^r} + \frac{a_1}{(s-\lambda)^{r-1}} + \cdots + \frac{a_{r-1}}{s-\lambda} + \frac{K_1}{s-\alpha_1} + + \frac{K_2}{s-\alpha_2} + \cdots + + \frac{K_{N-r}}{s-\alpha_{N-r}} 
\]

The $K_i$ values are determined using the same residue method as before. The $a_i$ values are found using a variation on the residue method:

\[
a_i = \left. \frac{1}{i!}\frac{d^i}{ds^i}[(s-\lambda)^r f(s)] \right|_{s = \lambda}
\]

That is rather complicated so lets finish up with an example. Find the partial fraction expansion of

\[
f(s) = \frac{s+3}{(s+1)^2(s+2)}
\]

Using the pattern above we expand the terms as

\[
f(s) = \frac{K_1}{s+2} + \frac{a_0}{(s+1)^2} + \frac{a_1}{s+1}
\]
then find the residues

\[
K_1 = \left. \frac{s+3}{(s+1)^2} \right|_{s = -2} = 1
\]

\[
a_0 = \left. \frac{1}{0!}\frac{d^0}{ds^0}\left[\frac{s+3}{s+2}\right] \right|_{s = -1} = 2
\]

\[
a_1 = \left. \frac{1}{1!}\frac{d}{ds}\left[\frac{s+3}{s+2}\right] \right|_{s = -1} = \left. \frac{(s+2)(1) - (s+3)(1)}{(s+2)^2} \right|_{s = -1} = -1
\]

Thus the result is

\[
f(s) = \frac{1}{s+2} + \frac{2}{(s+1)^2} + \frac{-1}{s+1}
\]

As you might expect there are computational tools that do this job for you.

\begin{itemize}
\item Mathematica: \texttt{Apart} function
\item Matlab (using symbolic toolbox) and Maxima: \texttt{partfrac} function
\end{itemize}

You can use this to check work and help on homework, but you will need to be able to do this by hand on exams.

\end{document}

