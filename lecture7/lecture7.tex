\documentclass{article}

\usepackage{url}
\usepackage{amsmath,amssymb}
\usepackage{graphicx,svg}

\begin{document}
\title{Lecture 7\\ Complex Analysis II}
\author{C.L. Wyatt}
\date{\today}
\maketitle

In today's lecture we continue our introduction to complex analysis by defining analytic complex functions, the complex derivative, and looking closer at rational complex functions.

\section{Complex Derivatives}

A complex function $f(s)$ is \textit{analytic} in some domain $R\subset\mathbb{C}$ of the function is

\begin{itemize}
\item single valued, and
\item has a finite complex derivative $f^\prime(s) \equiv \frac{df}{ds}$ for all $s\in R$.
\end{itemize}

Let $f(s) = f(x + jy) = u(x,y) + jv(x,y)$. The function has a complex derivative if and only if

\[
\frac{\partial u}{\partial x} = \frac{\partial v}{\partial y}  
\]
and
\[
\frac{\partial u}{\partial y} = -\frac{\partial v}{\partial x}  
\]
These are called the Cauchy-Riemann conditions. If these conditions are met then the complex dericative is given by

\[
f^\prime(s) = \frac{\partial u}{\partial x} + j \frac{\partial v}{\partial x} = \frac{\partial v}{\partial y} - j \frac{\partial u}{\partial y}
\]

If a complex function is analytic over the entire complex plane, i.e. $R = \mathbb{C}$, it is called an \textit{entire} function.

If a complex function is not analytic at a point $s_0$, but is analytic in the neighborhood of $s_0$, then the point is called a \texit{singularity} or a \textit{singular point} of the function.

\subsection{Examples}

\textbf{Example 1:} Let $s = x+jy$ and define

\[
f(s) = e^s = f(x+jy) = e^{x+jy} = e^x e^{jy}
\]
Using Euler's relation

\[
f(x+jy) = \underbrace{e^x\cos(y)}_{u(x,y)} + j\underbrace{e^x \sin(y)}_{v(x,y)}
\]
Checking the Cauchy-Riemann conditions

\[
\frac{\partial u}{\partial x} = e^x\cos(y)
\]

\[
\frac{\partial u}{\partial y} = -e^x\sin(y)
\]

\[
\frac{\partial v}{\partial x} = e^x\sin(y)
\]
\[
\frac{\partial v}{\partial y} = e^x\cos(y)
\]
Thus $\frac{\partial u}{\partial x} = \frac{\partial v}{\partial y} = e^x\cos(y)$ and $\frac{\partial u}{\partial y} = -\frac{\partial v}{\partial x} = -e^x\sin(y)$ and the function is analytic everywhere and thus an entire function. The complex derivative is

\[
f^\prime(s) = e^x\cos(y) + je^x\sin(y) = e^x e^{jy} = e^{x+jy} = e^s
\]
just as if $s$ was real and $f$ a real function.

\textbf{Example 2:} Let $f(s) = \frac{1}{s}$, for $|s| > 0$, i.e. $s \neq 0$. Then

\[
f(s) = f(x+jy) = \frac{1}{x+jy} = \frac{1}{x+jy} \cdot \frac{x-jy}{x-jy} = \frac{x-jy}{x^2 + y^2} = \underbrace{\frac{x}{x^2 + y^2}}_{u(x,y)} + j \underbrace{\frac{-y}{x^2 + y^2}}_{v(x,y)}
\]

Checking the Cauchy-Riemann conditions

\[
\frac{\partial u}{\partial x} = \frac{-x^2 + y^2}{(x^2 + y^2)^2}
\]

\[
\frac{\partial u}{\partial y} = \frac{-2xy}{(x^2 + y^2)^2}
\]

\[
\frac{\partial v}{\partial x} = \frac{2xy}{(x^2 + y^2)^2}
\]

\[
\frac{\partial v}{\partial y} = \frac{-x^2 + y^2}{(x^2 + y^2)^2}
\]

we we see that the function is analytic for $s \in \mathbb{C} - (0 + j0)$ and

\[
f^\prime(s) = \frac{-x^2 + y^2}{(x^2 + y^2)^2} + j \frac{2xy}{(x^2 + y^2)^2}
\]

With some work you can show this is the same as

\[
f^\prime(s) = -\frac{1}{s^2}
\]

again, just as if $s$ was real and $f$ a real function.

\textbf{Example 3:} Now for a counter-example. Consider

\[
f(s) = s^* = f(x + jy) = x - jy
\]
Thus $u(x,y) = x$ and $v(x,y) = -y$. Checking the Cauchy-Riemann conditions,

\[
\frac{\partial u}{\partial x} = 1
\]

\[
\frac{\partial u}{\partial y} = 0
\]

\[
\frac{\partial v}{\partial x} = 0
\]

\[
\frac{\partial v}{\partial y} = -1
\]

we see $\frac{\partial u}{\partial x} = 1 \neq \frac{\partial v}{\partial y} = -1$ thus the function is not analytic and no derivative exists.

\textbf{Example 4:} Show that $s_0 = -1$ is a singularity of
\[
f(s) = \frac{1}{s+1} = f(x +jy) = \frac{1}{(1+x) + jy}
\]

\end{document}

